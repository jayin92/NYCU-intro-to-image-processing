\documentclass[twocolumn]{extarticle}
\usepackage{fontspec}   %加這個就可以設定字體
\usepackage{xeCJK}       %讓中英文字體分開設置
\usepackage{indentfirst}
\usepackage{listings}
\usepackage[newfloat]{minted}
\usepackage{float}
\usepackage{graphicx}
\usepackage{caption}
\usepackage{fancyhdr}
\usepackage{hyperref}
\usepackage{amsmath}
\usepackage{multirow}
\usepackage[dvipsnames]{xcolor}
\usepackage{graphicx}
\usepackage{tabularx}
\usepackage{booktabs}
\usepackage{caption}
\usepackage{subcaption}
\usepackage{pifont}
\usepackage{amssymb}
\usepackage{titling}

\usepackage{pdftexcmds}
\usepackage{catchfile}
\usepackage{ifluatex}
\usepackage{ifplatform}

\usepackage[breakable, listings, skins, minted]{tcolorbox}
\usepackage{etoolbox}
\setminted{fontsize=\footnotesize}
\renewtcblisting{minted}{%
    listing engine=minted,
    minted language=python,
    listing only,
    breakable,
    enhanced,
    minted options = {
        linenos, 
        breaklines=true, 
        breakbefore=., 
        % fontsize=\footnotesize, 
        numbersep=2mm
    },
    overlay={%
        \begin{tcbclipinterior}
            \fill[gray!25] (frame.south west) rectangle ([xshift=4mm]frame.north west);
        \end{tcbclipinterior}
    }   
}

\usepackage[
top=1.5cm,
bottom=1.5cm,
left=1.5cm,
right=1.5cm,
includehead,includefoot,
heightrounded, % to avoid spurious underfull messages
]{geometry} 

\newenvironment{code}{\captionsetup{type=listing}}{}
\SetupFloatingEnvironment{listing}{name=Code}
\usepackage[moderate]{savetrees}


\title{Intro. to Image Processing HW1 Report}
\author{110550088 李杰穎}
\date{\today}


\setCJKmainfont{Noto Serif TC}


\ifwindows
\setmonofont[Mapping=tex-text]{Consolas}
\fi

\XeTeXlinebreaklocale "zh"             %這兩行一定要加,中文才能自動換行
\XeTeXlinebreakskip = 0pt plus 1pt     %這兩行一定要加,中文才能自動換行

\setlength{\parindent}{2em}
\setlength{\parskip}{2em}
\renewcommand{\baselinestretch}{1.25}
\setlength{\droptitle}{-10em}   % This is your set screw
\setlength{\columnsep}{2em}

\begin{document}

\maketitle

\section{Method}

在本次作業中,我使用了 \texttt{cv2} 和 \texttt{numpy} 來對圖片進行操作。

\subsection{Exchange Position}

使用雙層迴圈和 slice 的方法去將兩個二維陣列的部份內容互相交換。其中複製使用 shallow copy (\texttt{.copy()}) 而不是預設的 deep copy,使數值不會因為被修改而一同影響其他變數。

\subsection{Gray Scale}

灰階即是將 R, G, B 的數值取平均,要注意的是,因為 \texttt{cv2} 讀取圖片後的儲存資料型態是 \texttt{np.uint8},所以如果直接對陣列中的內容相加會導致 overflow。必須先轉成 Python 中的 int 型態再進行運算才不會出現 overflow 的問題。

\subsection{Intensity Resolution}

我們先將圖片轉為灰階,這個步驟和前面相同。之後在透過取底的運算,先將灰階的數值除以 64 再取底,這樣算出來的數值一定是 0, 1, 2, 3 其中一個。之後再將這個數字乘以 64 就可以產生 Intensity Resolution 為 4 的灰階圖片。具體計算公式可以參考下面式子。

\begin{equation}
	\text{value} = \left\lfloor\frac{\frac{R+G+B}{3}}{64}\right\rfloor \times 64
\end{equation}

\subsection{Red Color Filter}

一樣使用雙層迴圈去判斷該格的顏色是否符合投影片上的條件,如果沒有的話就轉成灰階。要注意的是透過 \texttt{cv2} 讀取的圖片,其 index 0, 1, 2 分別為 B, G, R。並非正常的 R, G, B,撰寫程式碼的時候需要特別小心。也要記得先將 \texttt{np.uint8} 轉換成內建的 int 格式。

\subsection{Yellow Color Filter}

與紅色濾鏡類似,不再贅述。

\subsection{Channel Operation}

綠色是對應到 index 1,故將每一像素的 index 1 乘以 2。但因為這樣的作法可能會造成 overflow 的問題,故在補上 \texttt{min(255, val*2)} 將數值限制在 255 內。

\subsection{Bilinear Interpolation}

雙線性和投影片和上課講授的相同,即是將最接近的四個點,先對 x 軸進行線性插值,分別插出兩個值。在用這兩個插值出的值,對 y 軸進行最後一步的線性插值,得到最後的結果。\footnote{此處提到的 x 軸為向下,y 軸則為向右。}

\subsection{Bicubic Interpolation}

雙三次插值則是透過最近的 16 個點,透過近似的方式求導,並解出多條三次函數,再透過此三次函數插值出中間的值。具體方式為先對 x 軸進行線性插值,產生四個三次函數,並產生四個相對應的數值。再利用這四個相對應的數值解出一條三次函數,最後再對 y 軸進行插值,得到最後的數值。要注意的是,不同於線性插值,三次插值的方式可能會使插值出來的數值不借於 0 ~ 255 間,導致 overflow。故會需要使用 \texttt{np.clip()} 函數,將陣列中的數值限制在 0 到 255 間。


\section{Result}

\autoref{fig:result} 即為最終的結果。

\begin{figure}[H]
	\centering
	\includegraphics[width=0.9\linewidth]{../result}
	\caption{透過 Python 程式碼所產生的結果圖}
	\label{fig:result}
\end{figure}


\section{Feedback}

在本次作業中,我了解如何透過 \texttt{cv2} 和 \texttt{numpy} 操作圖片,也學到了 bicubic 這個常見的插值方式具體是怎麼運作的,也對 Python 更加熟悉。

\end{document}